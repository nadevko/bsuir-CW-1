\chapter{Применение программы}

\section{Руководство пользователя}

\subsection{Установка программы}

Поддерживается установка как пакета, используя пакетный менеджер \textit{nix},
дистрибутива \textit{NixOS}. Для этого необходимо выполнить следующие шаги:

\begin{enumerate}
    \item Получить код из репозитория на \textit{GitHub}:
          \url{https:://github.com/nadevko/bsuir-CW1}. Для этого выполнить
          команду: \texttt{git clone https://github.com/nadevko/bsuir-CW1}.
    \item Перейти в директорию с проектом: \texttt{cd bsuir-CW1}.
    \item Установить пакет: \texttt{nix-env -if .}. Он будет скомпилирован и добавлен
          в систему.
\end{enumerate}

Альтернативно, можно использовать \textit{nix-shell} для запуска программы в
изолированной среде. Для этого требуется иная завершающая команда:
\texttt{nix-\\build -E~'with import <nixpkgs> \{ \}; callPackage}.

\subsection{Справка по использованию}

Программе передается список файлов и каталогов

Программа поддерживает следующие опции командной строки:

\begin{itemize}
    \item \texttt{-h, --help} -- выводит справку по использованию;
    \item \texttt{--help-all} -- выводит все параметры справки;
    \item \texttt{--help-gapplication} -- показывает параметры
          \textit{GApplication};
    \item \texttt{--help-hash} -- выводит справку по параметрам хеширования;
    \item \texttt{-v, --version} -- выводит версию программы;
    \item \texttt{-r, --recursive} -- осматривает каталоги рекурсивко;
    \item \texttt{-m, --mode} -- определяет режим работы программы;
    \item \texttt{-e, --export} -- определяет формат вывода результата;
    \item \texttt{-s, --side} -- устанавливает длину строки;
    \item \texttt{-b, --bins} -- размер интервалов;
    \item \texttt{--sigma} -- стандартное отклонение размытия по Гауссу;
    \item \texttt{--deviation} -- значение стандартного отклонения;
    \item \texttt{--median} -- значение отклонения медианы.
\end{itemize}
