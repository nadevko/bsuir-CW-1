\documentclass[variant=practice]{bsuir}

\departmentlong{инженерной психологии и эргономики}
\worktitle{ознакомительной}
\group{310901}
\departmentshort{ИПиЭ}
\student{А.Н. Бородин}
\manager{Д.А. Пархоменко}
\titlepageyear{2024}

\begin{document}

\maketitle

\chapter*{План прохождения практики}

\begin{enumerate}
      \item Организационное собрание: ознакомление с целями, задачами и
            содержанием практики. Инструктаж.
      \item Изучение вводного курса \textquote{Дизайнер интерфейсов} для
            специальности 6"=05"=0612"=01 Программная инженерия (профилизация:
            Инженерно"=психологическое обеспечение информационных технологий).
      \item Изучение профессиональных ролей в сфере информационных и
            коммуникационных технологий (ИКТ) и построение своего персонального
            пути карьеры (творческое задание).
      \item Экскурсия.
\end{enumerate}

\maketoc*

\chapter*{Введение}

\hyphenation{by-tes-pa-ce}

Ознакомительная практика на первом курсе обучения является неотъемлемой частью
учебного процесса, направленной на интеграцию теоретических знаний и
практических навыков. Она помогает студентам специальностей
\textquote{Программная инженерия} и \textquote{Информационные системы и
технологии} лучше понять свою будущую профессиональную деятельность и закрепить
полученные теоретические знания на практике.

В ходе практики был посещен посетили музей ретрокомпьютеров
\textit{\textquote{BYTESPACE}} при Минском Радиотехническом Колледже, где был
ознакомлен с историей развития информационных технологий и увидел, как с
течением времени менялись вычислительные системы. Этот опыт призван дать
представление о корнях современных технологий и вдохновить на осознанное
самостоятельное изучение дисциплин, как общенаучных, так и специализированных.

Целью практики служит овладение студентами практическими навыками, необходимыми
для успешной профессиональной деятельности. В ходе практики студенты изучают
организацию производства, задачи и функционирование предприятий и организаций, а
также приобретают навыки работы с нормативно"=технической и конструкторской
документацией.

С завершением практики студенты должны обладать знаниями о требованиях
безопасности, организационных основах производства, структуре управления и
основных процессах деятельности предприятий. Они также должны уметь применять
современные компьютерные технологии для обработки научно"=технической информации
и оформления отчета по практике, описывающего процесс и достижения практики, а
также овладеть методами ее поиска, обобщения и анализа, используя бумажные и
электронные носители, учебную, справочную и научно"=техническую литературу,
конструкторскую документацию предприятий, что сослужит в дальнейшем деятельности
при оформлении любого иного сообщения, реферата или презентации
\cite{about-practice}.

\chapter{Вводный курс \textquote{Дизайнер интерфейсов}}

Среди предложенных, наиболее любопытным был курс от Яндекса \cite{about-yandex},
делающий упор не на многочасовой просмотр видеороликов, а тестирования, игры и
практику. Курс не велик, представляет собой неплохое, хоть и поверхностное,
введение в базовые принципы дизайна интерфейсов для новичков. Для имеющих даже
минимальный опыт он бесполезен, из"=за чего оставил негативное впечатление. Далее
будет приведена краткая сводка по каждому уроку тем вводной части курса.

\section{Вступление}

\subsection{Урок 1.1} Дизайн -- это гармония, удобство, устройство чего"=либо,
внешний вид, систематизация информации и то, что придаёт объекту ценность.
Рекламируется платный курс.

\subsection{Урок 1.2} Специализаций дизайнеров великое множество: наружная
реклама, упаковка продуктов, постеры, стикеры, посты в соцсетях, сами
соцсети\dots Графический дизайнер помогает создавать визуальные образы, а
коммуникационный упаковывает их и адресует получателю.

\subsection{Урок 1.3} Дизайнеры интерфейсов отличаются от смежных им
веб"=дизайнеров, тем, что работают (дизайн поведения, а не только внешнего вида)
и продуктовых дизайнеров (меньшая зона ответственности), дали определение
\textit{UX/UI}.

\subsection{Урок 1.4} Описали чем дизайнер интерфейсов отличается от смежных
веб"=дизайнеров (дизайн поведения, а не только внешнего вида) и продуктовых
дизайнеров (меньшая зона ответственности), дали определение \textit{UX/UI}.

\subsection{Урок 1.5} \textit{hard skills} (исследование аудитории, создание
макетов, тестирование гипотез, аналитические способности) и \textit{soft skills}
(умение общаться, способность ставить себя на место другого человека и быть
гибким, умение аргументировать, готовность признавать ошибки и учиться новому)
дизайнера интерфейсов.

\subsection{Урок 1.6} Дополнительные навыки: аналитика конверсии (количественный
показатель, который отражает, какой процент пользователей, пришедших на сайт,
совершает то или иное целевое действие) на базе метрик, копирайт и работа с
адаптивным дизайном.

\subsection{Урок 1.7} Виды интерфейсов: сайты, мобильные приложения, автоматы и
терминалы, панели автомобилей, бытовая техника. Показатель успеха -- продуктом
пользуется всё больше и больше людей.

\subsection{Урок 1.8} Инструментарий дизайнера интерфейсов: \textit{Photoshop},
\textit{Sketch}, \textit{Adobe XD} и \textit{Figma}.

\subsection{Урок 1.9} Базовые функции и сочетания клавиш в \textit{Figma}.

\section{Магазин снов \textit{Dream Shop}}

\subsection{Урок 2.1} Лирическое отступление.

\subsection{Урок 2.2} Пользовательский сценарий (юзер"=флоу) -- путь
пользователя к целевому действию на сайте от точки входа до целевого действия;
\textit{CTA}"=кнопки (\textit{call-to-action}) и задание ее раскрасить (рисунок
\ref{img:yandex-cta.png}).

\subsection{Урок 2.3} Примеры поведенческих паттернов, упрощающих интерфейс, как
лайки, свайпы, скевоморфизм (иллюстрация реальным объектом), обманывающих темных
паттернов.

\subsection{Урок 2.4} Консистентность, простота, самоочевидность и уменьшение
работы пользователя как критерии юзабилити и юзерфрендли.

\subsection{Урок 2.5} Лирическое отступление.

\subsection{Урок 2.6} Сайт состоит из типовых страниц с разными целями и
структурой. Дизайнер помогает пользователю перейти с одной страницы на другую.

\subsection{Урок 2.7} Правила анализа юзер"=флоу: продумать каждый шаг сценария,
не забывать про существующие паттерны, придерживаться принципов юзабилити,
сделать заметный и однозначный \textit{CTA}.

\makeimage[\textit{CTA}"=кнопка]{yandex-cta.png}

\section{Простота и консистентность}

\subsection{Урок 3.1} Лирическое отступление.

\subsection{Урок 3.2} Главная страница сайта обязана наглядно демонстрировать,
что это за сайт и зачем он нужен.

\subsection{Урок 3.3} На страницах выделяют хедер (шапку) и футер (подвал);
часто встречаемые, сквозные элементы объединяют в сеты иконок и
\textit{UI}"=киты элементов интерфейса. Принято уделять много внимания логотипу
и акцентирующим цветам. Задание доделать карточку (рисунок
\ref{img:yandex-cards.png}).

\subsection{Урок 3.4} Меню в шапке и подвале -- основа навигации.

\subsection{Урок 3.5} Z"=паттерн -- люди, просматривая сайты, описывают взглядом
букву Z. Базовый принцип композиции.

\makeimage[Сетка карточек]{yandex-cards.png}

\subsection{Урок 3.6} Лирическое отступление.

\subsection{Урок 3.7} Элементы часто располагают по сетке, для соблюдения
правила близости.

\subsection{Урок 3.8} От карточки товара требуется название, цена и изображение.
Можно добавить лейбл.

\subsection{Урок 3.9} Итоги темы.

\section{Самоочевидность и обратная связь}

\subsection{Урок 4.1} Лирическое отступление.

\subsection{Урок 4.2} Люди не терпеливы -- нужно демонстрировать анимациями, что
сайт реагирует на действия.

\subsection{Урок 4.3} Состояния кнопок: обычное, ховер, нажатие, неактивное.

\subsection{Урок 4.4} Все элементы следуют оживлять по аналогии с кнопками.

\subsection{Урок 4.5} Лирическое отступление.

\subsection{Урок 4.6} (Хлебные) крошки -- цепочка ссылок на родительские разделы
сайта, которые дают возможность вернуться в предыдущий раздел. Лоадер, он же
крутилка или спиннер, оживляют малые процессы загрузки, прогресс"=бары для
крупных. Задания: сделать анимированный лоадер и прогресс"=бар (рисунки
\ref{img:yandex-loader.png} и \ref{img:yandex-progress.png} соответственно).

\makeimage[Анимированный лоадер]{yandex-loader.png}[width=0.65\textwidth]
\makeimage[Прогресс"=бар]{yandex-progress.png}[width=0.65\textwidth]

\subsection{Урок 4.7} С анимациями пользователю будет комфортнее.

\subsection{Урок 4.8} Итоги темы.

\section{Уменьшение работы пользователя}

\subsection{Урок 5.1} Лирическое отступление.

\subsection{Урок 5.2} Если пользователь не может что"=то найти, то виновен
дизайнер.

\subsection{Урок 5.3} Страница должна приводить всю информацию, что от нее
ожидает пользователь.

\subsection{Урок 5.4} Плохо спроектированная форма -- один из самых частых
камней преткновения на пути пользователя. Заполняя её, человек проходит через
все стадии признания неизбежности: отрицание, гнев, торг, депрессия и принятие.

\subsection{Урок 5.5} Хорошая форма не нагружает пользователя, удобно
располагает поля, оставляет подсказки и пометки обязательности поля, сообщает об
ошибках и помогает их исправлять, имеет кнопку подтверждения в конце и имеет
автозаполнение.

\subsection{Урок 5.6} Лирическое отступление.

\subsection{Урок 5.7} Плейсхолдер -- подсказка внутри поля ввода, тултип
является подсказкой всплывающей, контекстные подсказки появляются по триггеру, а
маски это интеллектуальные плейсхолдеры.

\subsection{Урок 5.8} Итоги темы.

\subsection{Урок 5.9} Итоги вводного курса.

\chapter{Путь карьеры}

Среди 30 ролей ИКТ специалистов \cite{majors}, сейчас выбор остановился на
эксперте \textit{DevOps}, но рассматривается и роль разработчика.

\section{Причины выбора}

\subsection{Отличное знание \textit{Linux}} Имеется многолетний опыт работы с
\textit{Linux}, включая сложные дистрибутивы, такие как \textit{Gentoo},
\textit{NixOS} и \textit{Arch}, что гарантирует отличные навыки работы с этой
операционной системой. Кроме того, речь не только об использовании систем, но и
об написания пакетов и модулей для них.

\subsection{Профессиональный опыт} В ходе попытки разработать свой веб"=сайт был
полностью настроен \textit{CI/CD} конвеер: \textit{Webpack}, \textit{PostCSS},
\textit{TypeScript}, минификация, \textit{CDN}, \textit{git}"=хуки для валидации
кода, используя \textit{ESLint} и \textit{CSSLint}, автоматическое развертывание
обновлений с помощью \textit{GitHub Actions} на \textit{netlify}, так что, опыт
работы по специальности уже есть.

\subsection{Высокая зарплата} Благодаря высокому спросу на квалифицированных
специалистов, по данным на \textquote{Хабр Карьера} \cite{devops-habr}, зарплата
в этой сфере в 1,36 раза выше средней в \textit{IT}"=индустрии \cite{devops-mts}.

\subsection{Востребованность} Спрос превышает предложение и должен был удвоиться
за последние пару лет, с 2022 к 2024 \cite{devops-mts}.

\subsection{Универсальность} \textit{DevOps} -- это смежная дисциплина,
требующая, помимо специфичных ей навыков, знаний в различных областях, включая
разработку, тестирование и безопасность, что способствует профессиональному
росту и дает возможность, при желании, брать на себя иные роли
\cite{devops-mts}.

\section{\textit{Hard skills}}

\subsection{Понимание основ \textit{DevOps}} Понимание принципов и практик
\textit{DevOps}, таких как непрерывная интеграция (\textit{CI}), непрерывная
доставка (\textit{CD}), автоматизация процессов, мониторинг и логирование.

\subsection{Программирование} Знание хотя бы одного скриптового языка, такого
как \textit{Python}, \textit{Ruby} или \textit{Go}, и понимание кода на
уровне, позволяющем вносить изменения и автоматизировать процессы
\cite{devops-mts}.

\subsection{Системное администрирование} Глубокие знания операционных систем
(особенно \textit{Linux}), управление файловыми системами, конфигурация сетей
\cite{devops-mts}.

\subsection{Инструменты \textit{CI/CD}} Опыт работы с инструментами непрерывной
интеграции и доставки, такими как \textit{Jenkins}, \textit{Travis CI},
\textit{GitLab CI}.

\subsection{Контейнеризация и оркестрация} Практический опыт с \textit{Docker} и
\textit{Kubernetes} для создания, развертывания и управления
контейнеризированными приложениями.

\subsection{Инфраструктура как код (\textit{IaC})} Использование инструментов,
таких как \textit{Terraform} и \textit{Ansible}, для автоматизации развертывания
и управления инфраструктурой.

\subsection{Облачные сервисы} Знание и опыт работы с облачными платформами,
такими как \textit{AWS}, \textit{Azure} и \textit{GCP}, включая их сервисы и
\textit{API}.

\subsection{Безопасность} Понимание основ кибербезопасности, умение
реализовывать безопасные решения и практики в рамках \textit{DevOps}"=процессов.

\subsection{Мониторинг и логирование} Знание инструментов и практик, таких как
\textit{EFK} и \textit{Zabbix}, для мониторинга состояния систем и приложений, а
также для сбора и анализа логов \cite{devops-mts}.

\subsection{Английский язык} Нужно знать технический английский, как и для любой
из 30 ролей \cite{devops-mts}.

\section{\textit{Soft skills}}

\subsection{Проактивность} Не выполнять поручения, а искать проблемы
самостоятельно \cite{devops-mts}.

\subsection{Работа в команде} Работа в команде: Умение эффективно сотрудничать с
другими членами команды, разделять задачи и поддерживать позитивную рабочую
атмосферу \cite{devops-mts}.

\subsection{Коммуникативные навыки} Способность ясно и чётко общаться как устно,
так и письменно.

\subsection{Эмпатия и понимание} Эмпатия и понимание проблем и перспектив других
людей.

\subsection{Решение проблем} Решение проблем: способность анализировать сложные
ситуации, находить творческие решения и принимать обоснованные решения.

\subsection{Адаптивность} Гибкость в принятии новых методов, технологий и
изменений в проектах.

\subsection{Управление временем} Эффективное планирование и приоритизация задач
для соблюдения сроков и достижения целей.

\subsection{Лидерские качества} Способность вести за собой, мотивировать команду
и принимать ответственность за проекты и решения.

\subsection{Обучаемость} Желание и способность быстро учиться и применять новые
знания в работе.

\section{Планы}

\subsection{Подготовиться к поиску работы} Для начала, следовало бы составить
резюме и зарегистрироваться на сайтах по поиску работы, как вариант, на
\url{rabota.by}.

\subsection{Начать работать} До сих пор, у меня не было опыта коммерческой
разработки. Следует получить хотя бы минимальный опыт, взявшись за любое, даже
самое маленькое задание.

\subsection{Заказы} Далее, приспособиться к выполнению заказов.
Справившись с первым заказом, не трудно будет выполнить и следующий. К концу
первого семестра второго курса следует хорошо выполнять их на постоянной основе.

\subsection{Найти постоянную работу} На втором курсе много кто начинает
работать. Было бы хорошо войти в это число. Уверенности в себе недостаточно,
чтобы считать этот пункт достижимым, из-за чего остановимся на первых трех.

\chapter{Экскурсия}

В рамках практики был посещен музей ретрокомпьютеров \textit{BYTESPACE}
\textquote{Минского Радиотехнического Колледжа}, с целью ознакомления с историей
развития вычислительной техники. Экскурсовод познакомил со старыми
калькуляторами (рисунок \ref{img:museum-calcs.jpg}), печатными машинками,
персональными компьютерами (рисунок \ref{img:museum-pc.jpg}), их клавиатурными
одноплатными моделями, ноутбуками, консолями и КПК, объединив их истории в
хронологическое повествование о развитии компьютерных технологий.

\makeimage[Калькуляторы]{museum-calcs.jpg}[width=0.55\textwidth]
\makeimage[Персональные компьютеры и игровая зона]{museum-pc.jpg}
[width=0.55\textwidth]

Этот музей имеет одну особенность -- все компьютеры поддерживаются в
работоспособном состоянии и посетителям разрешено ими пользоваться, даже
выделена игровая зона (рисунок \ref{img:museum-pc.jpg}) с ретроконсолями, в
которые можно сыграть. Однако, времени на это не хватило.

Среди экспонатов хотелось бы отметить единственный УВК (управляющий
вычислительный комплекс) СМ 1420, изображенный на рисунке
\ref{img:museum-blocks.jpg} и впечатливший своим аутентичным, напоминающим
современные серверные шкафы, видом, одноплатный ZX Spectrum (рисунок
\ref{img:museum-spectrum.jpg}) и руководство по бывшему на нем популярным языку
бейсик, которое удалось почитать и которое заставило задуматься о том, чтоы
любая ныне широко распространенная технология в будущем может быть единогласно
признана ошибкой и предана забвению.

\makeimage[СМ 1420]{museum-blocks.jpg}[width=0.53\textwidth]
\makeimage[\textit{ZX Spectrum 128k}]{museum-spectrum.jpg}[width=0.53\textwidth]

\chapter*{Выводы по результатам прохождения учебной (ознакомительной) практики}

Во"=первых, был пройден вводный курс, который ничего нового не дал. Может,
платная часть и полезна, но вводная полезна для новичков, не имеющих ни
малейшего опыта. Задания слишком просты (на каждое ушло, около трех минут), а
затрагиваемые темы поверхностны. Обобщая пройденное: дизайнеры интерфейсов
разрабатывают и оптимизируют юзер"=флоу (сценарии поведения пользователя),
опираясь на психологические эффекты, подобные Z"=паттерну (обычен осмотр
страницы, описывая глазами букву Z, проходя шапку, тело и подвал), скевоморфизму
(метафора через изображение реального объекта того же назначения), акцентам
(привлекающие внимание элементы, цветовое подчеркивание) и тотальной анимации
(если мгновенного отклика не было, человек думает, что что"=то пошло не так) с
целью помощи пользователям в работе с продуктом, а также повышения его
привлекательности и рентабельности.

Посещение музея ретрокомпьютеров \textit{BYTESPACE} вызвало интерес и придало
мотивации стараться, в надежде занять место в истории, но очень сильно не
хватило времени на личный осмотр экспозиции. Хотелось поглядеть на компьютеры
в работе и взглянуть на установленные на них программы.

В целом, не смотря на негативную оценку курса и экскурсии, практика оказалась
очень полезной. До творческого задания у меня не было четкого видения желаемой
роли и каких"=либо карьерных целей: его выполнение заставило задуматься над этим
и определиться к чему стремиться, а написание отчета помогло набраться опыта в
оформлении стандартизированных пояснительных записок.

\bibliography{lib}

\end{document}
