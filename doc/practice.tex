\documentclass[variant=practice]{bsuir}

\departmentlong{инженерной психологии и эргономики}
\worktitle{ознакомительной}
\group{310901}
\departmentshort{ИПиЭ}
\student{А.Н. Бородин}
\manager{Д.А. Пархоменко}

\begin{document}

\maketitle

\chapter*{План прохождения практики}

\begin{enumerate}
      \item Организационное собрание: ознакомление с целями, задачами и
            содержанием практики. Инструктаж.
      \item Изучение вводного курса \textquote{Дизайнер интерфейсов} для
            специальности 6-05-0612-01 Программная инженерия (профилизация:
            Инженерно"-психологическое обеспечение информационных технологий).
      \item Изучение профессиональных ролей в сфере информационных и
            коммуникационных технологий (ИКТ) и построение своего персонального
            пути карьеры (творческое задание).
      \item Экскурсия.
\end{enumerate}

\maketoc*

\chapter*{Введение}

Проводимая в конце первого курса учебная (ознакомительная) практика является
важным этапом обучения, который помогает студентам лучше понять свою будущую
профессию и закрепить теоретические знания на практике. Практика подготавливает
студентов к их профессиональной деятельности, ознакомливая с работой организаций
и предприятий, в которых они будут востребованы, организацией их производства,
задачами, оснащением и принципами функционирования. Побочно, практика
подготавливает студентов к самостоятельному осознанному изучению дисциплин, как
общенаучного, так и узкоспециализированного профессионального и учебного плана.

В ходе практики проводится исследование структур организаций Беларуси,
работающих в сфере информационных технологий, представляющее собой сбор
информации о них, в том числе, в ходе экскурсии, ее анализа и обобщения.
Студенты учатся применять свои общенаучные знания, подкрепляя ими знания
общепрофессиональные и специально"-дисциплинарные, обрабатывая собранные
материалы и работая с нормативно"-технической и конструкторской документацией, в
частности, с отчетом о прохождении практики, получают новые знания на
предприятиях и разбираются, как применить эту информацию в дальнейшей научной и
учебной деятельности.

С завершением, студенты считаются ознакомленными с требованиями безопасности при
посещении предприятия, как благополучно посетившие таковое и владеющие навыками
поведения на предприятии, ознакомленными с базовой организационние
производства, включающей структуры управления производством и посещенного отдела
информационных технологий, взаимоотношениями последнего с остальными
подразделениями организации, главными технологиями и программными средствами,
используемыми в отделе, и основными рабочими процессами предприятия. Студенты
должны владеть методами поиска, обобщения и анализа научно"-технической
информации, используя бумажные и электронные носители, учебную, справочную и
научно"-техническую литературу, конструкторскую документацию предприятия, быть
способны применять эти знания для оформления отчета по практике или любого иного
сообщения, реферата, презентации в дальнейшем\cite{about-practice}.

\chapter{Вводный курс \textquote{Дизайнер интерфейсов}}

Среди предложенных, наиболее любопытным был курс от Яндекса\cite{about-yandex},
делающий упор не на многочасовой просмотр видеороликов, а тестирования, игры и
практику. Курс не велик, но представляет собой неплохое, хоть и поверхностное,
введение в базовые принципы дизайна интерфейсов для новичков, но не для имеющих
минимальный опыт. Далее будет приведена краткая сводка по каждому уроку всех
тем вводной части курса.

\section{Вступление}

\subsection{Урок 1.1} Дизайн -- это гармония, удобство, устройство чего-либо,
внешний вид, систематизация информации и то, что придаёт объекту ценность.
Рекламируется платный курс.

\subsection{Урок 1.2} Специализаций дизайнеров великое множество: наружная
реклама, упаковка продуктов, постеры, стикеры, посты в соцсетях, сами
соцсети\dots Графический дизайнер помогает создавать визуальные образы, а
коммуникационный упаковывает их и адресует получателю.

\subsection{Урок 1.3} Дизайнеры интерфейсов отличаются от смежных им
веб"-дизайнеров, тем, что работают (дизайн поведения, а не только внешнего вида)
и продуктовых дизайнеров (меньшая зона ответственности), дали определение
\textit{UX/UI}.

\subsection{Урок 1.4} Описали чем дизайнер интерфейсов отличается от смежных
веб"-дизайнеров (дизайн поведения, а не только внешнего вида) и
продуктовых дизайнеров (меньшая зона ответственности), дали
определение \textit{UX/UI}.

\subsection{Урок 1.5} \textit{hard skills} (исследование аудитории, создание
макетов, тестирование гипотез, аналитические способности) и \textit{soft skills}
(умение общаться, способность ставить себя на место другого человека и быть
гибким, умение аргументировать, готовность признавать ошибки и учиться новому)
дизайнера интерфейсов.

\subsection{Урок 1.6} Дополнительные навыки: аналитика конверсии (количественный
показатель, который отражает, какой процент пользователей, пришедших на сайт,
совершает то или иное целевое действие) на базе метрик, копирайт и работа с
адаптивным дизайном.

\subsection{Урок 1.7} Виды интерфейсов: сайты, мобильные приложения, автоматы и
терминалы, панели автомобилей, бытовая техника. Показатель успеха -- продуктом
пользуется всё больше и больше людей.

\subsection{Урок 1.8} Инструментарий дизайнера интерфейсов: \textit{Photoshop},
\textit{Sketch}, \textit{Adobe XD} и \textit{Figma}.

\subsection{Урок 1.9} Базовые функции и сочетания клавиш в \textit{Figma}.

\section{Магазин снов \textit{Dream Shop}}

\subsection{Урок 2.1} Лирическое отступление.

\subsection{Урок 2.2} Пользовательский сценарий (юзер"-флоу) -- путь
пользователя к целевому действию на сайте от точки входа до целевого действия;
\textit{CTA}-кнопки (\textit{call-to-action}) и задание ее раскрасить (рисунок
\ref{img:yandex-cta.png}).

\makeimage[\textit{CTA}-кнопка]{yandex-cta.png}[width=0.8\textwidth]

\subsection{Урок 2.3} Примеры поведенческих паттернов, упрощающих интерфейс, как
лайки, свайпы, скевоморфизм (иллюстрация реальным объектом), обманывающих темных
паттернов.

\subsection{Урок 2.4} Консистентность, простота, самоочевидность и уменьшение
работы пользователя как критерии юзабилити и юзерфрендли.

\subsection{Урок 2.5} Лирическое отступление.

\subsection{Урок 2.6} Сайт состоит из типовых страниц с разными целями и
структурой. Дизайнер помогает пользователю перейти с одной страницы на другую.

\subsection{Урок 2.7} Правила анализа юзер"-флоу: продумать каждый шаг сценария,
не забывать про существующие паттерны, придерживаться принципов юзабилити,
сделать заметный и однозначный \textit{CTA}.

\section{Простота и консистентность}

\subsection{Урок 3.1} Лирическое отступление.

\subsection{Урок 3.2} Главная страница сайта обязана наглядно демонстрировать,
что это за сайт и зачем он нужен.

\subsection{Урок 3.3} На страницах выделяют хедер (шапку) и футер (подвал);
часто встречаемые, сквозные элементы объединяют в сеты иконок и \textit{UI}-киты
элементов интерфейса. Принято уделять много внимания логотипу и акцентирующим
цветам. Задание доделать карточку (рисунок \ref{img:yandex-cards.png}).

\makeimage[Сетка карточек]{yandex-cards.png}

\subsection{Урок 3.4} Меню в шапке и подвале -- основа навигации.

\subsection{Урок 3.5} Z-паттерн -- люди, просматривая сайты, описывают взглядом
букву Z. Базовый принцип композиции.

\subsection{Урок 3.6} Лирическое отступление.

\subsection{Урок 3.7} Элементы часто располагают по сетке, для соблюдения
правила близости.

\subsection{Урок 3.8} От карточки товара требуется название, цена и изображение.
Можно добавить лейбл.

\subsection{Урок 3.9} Итоги темы.

\section{Самоочевидность и обратная связь}

\subsection{Урок 4.1} Лирическое отступление.

\subsection{Урок 4.2} Люди не терпеливы -- нужно демонстрировать анимациями, что
сайт реагирует на действия.

\subsection{Урок 4.3} Состояния кнопок: обычное, ховер, нажатие, неактивное.

\subsection{Урок 4.4} Все элементы следуют оживлять по аналогии с кнопками.

\subsection{Урок 4.5} Лирическое отступление.

\subsection{Урок 4.6} (Хлебные) крошки -- цепочка ссылок на родительские разделы
сайта, которые дают возможность вернуться в предыдущий раздел. Лоадер, он же
крутилка или спиннер, оживляют малые процессы загрузки, прогресс"-бары для
крупных. Задания: сделать анимированный лоадер (рисунок
\ref{img:yandex-loader.png}) и прогресс"-бар (рисунок
\ref{img:yandex-progress.png}).

\makeimage[Анимированный лоадер]{yandex-loader.png}[width=0.4\textwidth]

\makeimage[Прогресс-бар]{yandex-progress.png}[width=0.4\textwidth]

\subsection{Урок 4.7} С анимациями пользователю будет комфортнее.

\subsection{Урок 4.8} Итоги темы.

\section{Уменьшение работы пользователя}

\subsection{Урок 5.1} Лирическое отступление.

\subsection{Урок 5.2} Если пользователь не может что"-то найти, то виновен
дизайнер.

\subsection{Урок 5.3} Страница должна приводить всю информацию, что от нее
ожидает пользователь.

\subsection{Урок 5.4} Плохо спроектированная форма -- один из самых частых
камней преткновения на пути пользователя. Заполняя её, человек проходит через
все стадии признания неизбежности: отрицание, гнев, торг, депрессия и принятие.

\subsection{Урок 5.5} Хорошая форма не нагружает пользователя, удобно
располагает поля, оставляет подсказки и пометки обязательности поля, сообщает об
ошибках и помогает их исправлять, имеет кнопку подтверждения в конце и имеет
автозаполнение.

\subsection{Урок 5.6} Лирическое отступление.

\subsection{Урок 5.7} Плейсхолдер -- подсказка внутри поля ввода, тултип
является подсказкой всплывающей, контекстные подсказки появляются по триггеру, а
маски это интеллектуальные плейсхолдеры.

\subsection{Урок 5.8} Итоги темы.

\subsection{Урок 5.9} Итоги вводного курса.

\chapter{Путь карьеры}
\chapter{Экскурсия}

\chapter*{Заключение}

\bibliography{common}

\end{document}
