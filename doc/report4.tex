\chapter{Тестирование}

Для тестирования программы применялся минималистичный \textit{GLib Testing
Framework}, являющийся частью \textit{GLib}, его \textit{GTK}"=расширение и
\textit{Meson}, предоставляющий интерфейс для запуска тестов. Эта библиотека
использоовалась в ходе разработки программы для проверки корректности работы и
сравнения результатов вычисления хеша с образцом, за который дзята реализация
хеширования из библиотеки \textit{OpenCV}:
\textit{cv::img\_hash::RadialVarianceHash}.

Тесты проводились на изображениях различных размеров и разрешений, а также на
изображениях с различными характеристиками, такими как цвет, форма и текстура, в
рамках рисунков к курсовой работе и случайный сохраненных изображений. Если
стандартное количество секторов радиального разбиения в \textit{OpenCV} и правда
равно 180, то возможно, данная реализация выдает лучшие результаты.

При сравнении изображений"=примеров, изображенных на рисунке \ref{odiff}, был
получен процент около 93 процентов, когда OpenCV показывал на 3 больше, за счет
чего можно утверждать об эффективность данной реализации.

Также, было выяснено, что при количестве секторов свыше 10 миллионов операция
сравнения дает сбой и выходит далеко за пределы требуемого диапазона.

\vspace{\baselineskip}
\phantomheading[section]{Заключение}

В разделе было описано тестирование программы, проведенное с частичным
использованием \textit{GLib Testing Framework}, для проверки корректности работы
и сравнения результатов вычисления хеша с образцом, за который дзята реализация
хеширования из библиотеки \textit{OpenCV}: \textit{RadialVarianceHash} из модуля
\textit{cv::img\_hash}.
