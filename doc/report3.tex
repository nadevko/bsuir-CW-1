
\chapter{Программная реализация}

\section{Физическая структура}

\subsection{src/assets}

\textit{src/assets} -- Каталог с ресурсами приложения.

\subsection{src/assets/logo}

\textit{src/assets/logo} -- Каталог с логотипами приложения.

\subsection{src/assets/logo/128x128.png}

\textit{src/assets/logo/128x128.png} -- Логотип приложения размером 128x128
пикселей.

\subsection{src/assets/logo/256x256.png}

\textit{src/assets/logo/256x256.png} -- Логотип приложения размером 256x256
пикселей.

\subsection{src/assets/logo/512x512.png}

\textit{src/assets/logo/512x512.png} -- Логотип приложения размером 512x512
пикселей.

\subsection{src/assets/logo/scalable.svg}

\textit{src/assets/logo/scalable.svg} -- Логотип приложения в векторном формате.

\subsection{src/assets/meson.build}

\textit{src/assets/meson.build} -- Файл сборки ресурсов.

\subsection{src/assets/template.desktop}

\textit{src/assets/template.desktop} -- Шаблон файла \textit{.desktop} для
создания ярлыка приложения.

\subsection{src/gresources.xml}

\textit{src/gresources.xml} -- Перечень ресурсов приложения для их загрузки в
приложении.

\subsection{src/tests}

\textit{src/tests} -- Каталог с модульными тестами.

\subsection{src/tests/meson.build}

\textit{src/tests/meson.build} -- Файл сборки модульных тестов.

\subsection{src/tests/main.cc}

\textit{src/tests/main.cc} -- Точка входа в модульные тесты.

\subsection{src/views}

\textit{src/views} -- Каталог с файлами пользовательского интерфейса.

\subsection{src/views/about.ui}

\textit{src/views/about.ui} -- Файл пользовательского интерфейса для окна
\textit{About}.

\subsection{src/views/application.ui}

\textit{src/views/application.ui} -- Файл пользовательского интерфейса для
главного окна приложения.

\subsection{src/views/popover.ui}

\textit{src/views/popover.ui} -- Файл пользовательского интерфейса для
всплывающего окна.

\subsection{src/views/preferences.ui}

\textit{src/views/preferences.ui} -- Файл пользовательского интерфейса для окна
настроек.

\subsection{src/views/project.cmb}

\textit{src/views/project.cmb} -- Файл проекта \textit{Cambalache}.

\subsection{src/main.cc}

\textit{src/main.cc} -- Точка входа в приложение. В нем инициализируются
\textit{GetText} и \textit{adwaita}, создается экземпляр
\textit{CW1::Application} и запускается само приложение.

\subsection{src/meson.build}

\textit{src/meson.build} -- Файл сборки приложения. В нем собирается
испольняемый файл, загружается \textit{GResource} и запускается сборка тестов и
ресурсов.

\subsection{src/rvhash.hh}

\textit{src/rvhash.hh} -- Объявление класса \textit{CW1::RVHash} для хеширования
(модуль хеширования).

\subsection{src/rvhash.cc}

\textit{src/rvhash.cc} -- Имплементация класса \textit{CW1::RVHash} для
хеширования (модуль хеширования).

\subsection{src/application.cc}

\textit{src/application.cc} -- Имплементация класса \textit{CW1::Application}
для инициализации приложения и для обработки опций командной строки (модуль
приложения).

\subsection{src/application.hh}

\textit{src/application.hh} -- Имплементация класса \textit{CW1::Application} для
инициализации приложения и для обработки опций командной строки (модуль
приложения).

\subsection{src/processor.hh}

\textit{src/processor.hh} -- Объявление абстрактных классов
\textit{CW1::Processor} для обхода файлов (модуль итерации) и
\textit{CW1::Processor::Formatter} для форматирования вывода (модуль действий).

\subsection{src/hashformatter.cc}

\textit{src/hashformatter.cc} -- Имплементация класса
\textit{CW1::HashFormatter}, расширяющего класс
\textit{CW1::Processor::Formatter}, для обхода файлов (модуль итерации).

\subsection{src/hashformatter.hh}

\textit{src/hashformatter.cc} -- Объявление класса
\textit{CW1::HashFormatter}, расширяющего класс
\textit{CW1::Processor::Formatter}, для форматирования вывода (модуль действий).

\subsection{src/hashprocessor.cc}

\textit{src/hashprocessor.cc} -- Имплементация класса
\textit{CW1::HashProcessor}, расширяющего класс \textit{CW1::Processor}, для
обхода файлов (модуль итерации).

\subsection{src/hashprocessor.hh}

\textit{src/hashprocessor.cc} -- Объявление класса \textit{CW1::HashProcessor},
расширяющего класс \textit{CW1::Processor}, для обхода файлов (модуль итерации).

\section{Описание разработанных модулей}

\subsection*{Модуль приложения}

\subsubsection{\textit{CW1::Application}}

\textit{CW1::Application} -- класс, наследуемый от \textit{Gtk::Application}. Он
олицетворяет собою сеанс приложения, запускается из \textit{main}.

Публичные методы класса:

\begin{itemize}
    \item \textit{Application} -- инициализация приложения, включение режима
          ручной обработки опций программы и объявления опций командной строки.
    \item \textit{create} -- статический метод, который создает новый экземпляр
          класса \textit{Application} и возвращает его в виде умного указателя
          \textit{Glib::RefPtr} (на самом деле, это \textit{std::shared\_ptr}).
          Создан по аналогии с оригинальным классом \textit{Gtk::Application}
          для унификации поведения.
\end{itemize}

Защищенные методы класса:

\begin{itemize}
    \item \textit{on\_command\_line} -- метод, вызываемый при консольном запуске
          программы. Он обрабатывает опции командной строки и, если требуется,
          выполняет методы.
    \item \textit{on\_activate} -- обработчик сигнала \textit{activate}
          приложения. Он выполняется только при запуске графической сессии.
\end{itemize}

Приватные поля класса:

\begin{itemize}
    \item \textit{optionsFilters} -- объект группы опций. Объединяет те, что
          связаный настройкой поведения хеш"=функции.
    \item \textit{side, bins, sectors, sigma, stdDeviationThreshold,
          medianThreshold, rvhash} -- хранят экземпляр \textit{rvhash} и его
          параметры (нельзя получить значение опции из группы, не присваивая ее
          полю класса).
\end{itemize}

\subsection*{Модуль хеширования}

\subsubsection{\textit{CW1::RVHash}}

\textit{CW1::RVHash} -- класс"=обертка, объединяющий в себе методы все методы
для вычисления хеша радиальной дисперсии (\textit{Radial Variance Hash}). Методы
полностью соответствуют шагам, описанным в разделе \ref{sec:2.1}. 

Публичные элементы класса:

\begin{itemize}
    \item \textit{hash} -- псевдоним типа \textit{uint64\_t}. Формат хеша.
    \item \textit{compute} -- Вычисляет типа \textit{hash} хеш переданного ему
          изображения в формате \textit{Gdk::PixBuf} из соответственной
          библиотеки.
    \item \textit{compare} -- сравнивает хеши и возвращает
          \textit{double}"=показатель сходства. При верной конфигурации хеша
          находится в диапазоне от 0 до 1 и аналогичен проценту шанса
          совпадения.
\end{itemize}

Защищенные методы класса:

\begin{itemize}
    \item \textit{on\_command\_line} -- метод, вызываемый при консольном запуске
          программы. Он обрабатывает опции командной строки и, если требуется,
          выполняет методы.
    \item \textit{on\_activate} -- обработчик сигнала \textit{activate}
          приложения. Он выполняется только при запуске графической сессии.
\end{itemize}

Приватные методы класса:

\begin{itemize}
    \item \textit{get\_scaled} -- возвращает изображение нового размера:
          квадрат, длина стороны которого равна \textit{size}.
    \item \textit{get\_grayscale} -- обесцвечивает изображение.
    \item \textit{get\_sector\_stats} --обертка для метода \textit{get\_median}
          и \textit{get\_stddeviation}, обрабатывающая нулевые значения.
    \item \textit{get\_median} -- возвращает медиану значений пикселей
          изображения.
    \item \textit{get\_stddeviation} -- возвращает стандартное отклонение
          значений пикселей изображения.
    \item \textit{get\_gaussian\_blur} -- возвращает изображение, размытое
          гауссовым фильтром.
    \item \textit{get\_gaussian\_kernel} -- возвращает ядро гауссова фильтра.
    \item \textit{get\_filtered} -- возвращает изображение, прошедшее фильтрацию
          по ядру.
\end{itemize}

Приватные поля, у каждого из которых есть свой сеттер:

\begin{itemize}
    \item \textit{size} -- длина стороны квадрата, в который преобразуется
          изображение.
    \item \textit{bins} -- интервалы, по аналогии с \textit{OpenCV}. Кажется,
          они нигде не используются.
    \item \textit{sectors} -- количество секторов.
    \item \textit{sigma} -- стандартное отклонение размытия по Гауссу.
    \item \textit{stdDeviationThreshold} -- пороговое значение стандартного
          отклонения.
    \item \textit{medianThreshold} -- порогое значение медианы.
\end{itemize}

\subsection*{Модуль итерации}

\subsubsection{\textit{CW1::Processor}}

\textit{CW1::Processor} -- абстрактный класс для обхода файловой системы и
вывода результатов, используя \textit{CW1::Processor::Formatter}. Тоже
использует внутри себя \textit{CW1::RVHash}, по"=этому дублирует весь набор
сеттеров и приватных полей параметров хеша.

Публичные элементы класса:

\begin{itemize}
    \item \textit{Table} -- псевдоним типа \textit{std::map<std::string,
          CW1::RVHash::hash>}. Таблица хешей.
    \item \textit{process\_filepaths} -- виртуальный метод, возвращает
          таблицу хешей.
    \item \textit{process\_file} -- виртуальный метож, который обрабатывает
          файл и добавляет хеш в таблицу хешей, передаваемую вторым аргументом.
    \item \textit{set\_recursive} -- сеттер рекурсивного режима.
    \item \textit{set\_rvhash} -- сеттер экземпляра \textit{CW1::RVHash}.
    \item \textit{set\_formatter} -- сеттер экземпляра
          \textit{CW1::Processor::Formatter}.
    \item \textit{to\_string} -- возвращает строку отфарматированную строку.
    \item \textit{is\_image\_file} -- проверяет, является ли файл изображением,
          поддерживаемым \textit{Gdk::PixBuf}, за счет попытки его создания.
\end{itemize}

\subsubsection{\textit{CW1::HashProcessor}}

\textit{CW1::HashProcessor} -- реализация абстрактного класса, правила обхода:
для каждого каталога, обходит его содержимое, каталоги внутри него, только если,
указан флаг \textit{recursive}. Проверяет все файлы, являются ли они
изображениями. В полученний таблице будут вектора с одним элементом -- хешем
этого файла.

Определяет виртуальные методы \textit{process\_filepaths} и
\textit{process\_file}. Добавляет защищеный метод \textit{process\_directory},
реализующий рекурсию.

\subsection*{Модуль действий}

\subsubsection*{\textit{CW1::Processor::Formatter}}

Определяет только публичный виртуальный метод \textit{format}, который принимает
таблицу хешей и поток вывода, в который он записывает результаты.

\subsubsection*{\textit{CW1::HashFormatter}}

Реализует вывод хешей в форме: \textquote{путь к файлу | хеш}. Вывод
отфарматирован так, что образуется цельтая вертикальная черта.

\vspace{\baselineskip}
\phantomheading[section]{Заключение}

В разделе были составлены перечни файлов и описаны разработанные модули. Для
каждого класса и метода дана краткая сводка, упомянуты детали реализации и
расписаны отношения между классами. Более точную информацию можно найти в
диаграмме классов, представленной в приложении \ref{sec:В}. Ее точность
гарантируется тем, что она была сгенерирована автоматически, используя утилиты
\textit{clang"=uml} и \textit{plantuml}.
